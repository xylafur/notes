Composition:
    2)  How does each painter create the illusion of three-dimensional form and
        space?  Is one painting relatively more “three-dimensional” than the
        other? Why does it have this effect?

        Romano:
            From the placment of the figures - Jesus in front with mary in the
            back.  Her arm over the child and Jesus's arm wrapped around the
            back of the virgin's neck

        Dolci:
            The light source and shading, way edges are depicted


    3)  In which painting does the artist think of discrete, individual forms,
        and in which is there a greater “unity” of forms—i.e., so that the
        figures, limbs, etc. are more closely and dynamically intertwined with
        each other?

        Romano:
            Entities are each an individual part, 

        Dolci:


    4)  In which painting is the composition completely contained by the frame,
        and in which does it seem to extend beyond the borders of the frame?
        How does this affect our perception of the relation between “our”
        space—the viewer’s space—and the space of the picture?

        Romano:
            Completley in the frame

        Dolci:
            The virgin as well as John the Baptist extent somewhat beyond the
            edge of the frame

            This depiction, along with the greater shading and level of detail
            in the painting create the illusion that the figures are more in
            our space than in some painting


Emotional Level:
    1)  How do the Virgin and Child relate to each other and to the viewer?
        You may focus on their poses, gestures, facial expressions, gazes, etc.

        Romano:

        Dolci:


    2)  Note the figure of St. John the Baptist in Dolci’s painting:  do you
        see any parallels in how it functions within the scene to the figure of
        the donor in Romano’s picture?

        Romano:

        Dolci:


    3)  In general, how do these paintings compare to each other in their
        expressive, emotional content?  Is one, relatively, more “restrained”
        and the other more intensely emotional than the other?

        Romano:

        Dolci:


Romano:

Dolci:


Paper Structure:
    Thesis:
        Throughout the course of the Renisance, there was an emphasis first on
        naturalism in art and then on dramatic play of light and dark /
        emotions as the idea of the artist became more intwined with that of
        intellectuality and sophistication while the creation of man was
        compared to divine creation.

        The overall expressiveness of art increased dramatically from the
        beginning of the Renaissance to the Baroque period

    Paragraph 1: Visual Charactersitics - structure of the figures
        Start with the similarities between the two paintings.  They both
        obviously depict the same iconography.  Additionally the overall
        structure of the paintings is the same (how the figures are arranged).

        Talk about how in the one painting the figures are more tied together
        and much more dynamic than in the other.  Lead in to how the overall
        light and dark effects plays a huge role in this

    Paragraph 2: Visual Characteristics - Color and light

        Then get to the differences: the greater play of colors, the dynamic
        figures

    Paragraph 3: Composition of the figures - dynamic vs static





    Paragraph 3:
        




