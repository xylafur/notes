Visual Characteristics

    1)  Which painter thinks more in terms of precise edges to his forms, and
        which allows his figures and objects to have softer contours, so that
        they seem to fade into the shadowed background?

        Romano:
            Romano uses much more percise edges in all of the figures.  We can
            clearly see where one figure ends and the backgound or another
            figure begins.

        Dolci:
            Dolci uses very soft edges that seem to blend with their
            surroundings. Where one figure ends and another begins is much more
            explicit, though the distinction between figures it is obvious; the
            painting is much more naturalistic in this sense.

    2) Describe the quality and use of color by each artist:  does one painting
       contain more intense colors than the other?

        Romano:
            The color set in this painting is very dull and flat.  The colors
            appear to be accurate, though there is not a significant level of
            naturalism in them.  The colors are very solid.  Each different
            object seems to have a particular color.  FOr instance the virgin's
            cloak is all a solid dark blue and her skin is all the same pale
            peach.

        Dolci:
            The colors in this painting are much more significant then in the
            Romano.  The colors seem to scream at the viewer - not only are
            they very vibrant but they are very dynamic; no object is made of
            just a single color, each object varies in absolute color from one
            location to another on the object.  A significant part of this
            stems from how the artist choses to depict lighting.

            THe large expression of color can be seen in the difference between
            the rosy cheeks of all of the figures and the way shadows have
            effects on the absolute color of both skin and clothes.

    3)  Does one painting have greater contrast in value (light and dark) than
        the other?  How do these qualities of color and light/dark contrast
        affect the overall visual effect of the paintings?

        Romano:
            There appears to be only a single light source that does not cast
            shadows in this painting (other than the faint shadows on the floor
            of the table on which the child is standing).  THis lack of
            shadowing makes the painting appear to be flat even though the
            figures are depicted with 3 dimentional form.

            No section of the painting really appears to be dark

        Dolci:
            There is a significant difference between light and dark in this
            image.  The figures appear to fade in to darkness.  THis effect is
            amplified by the way in which edges are depicted (very softly).

            Shadows are very prominent in this painting, Not only making the
            clothes of the virgin appear to be garments on an actual figure but
            also making the entire image appear more three dimensional.  The
            way light and dark, colors and edges are depicted in this image
            make it look significantly life like.

            The halos are much more prominant due to the shadows as well.  THe
            contrast between the light above their heads and the darkness of
            the background makes them actually seem more divine.

    4)  Can you discern how the artist applied the paint to the support (wood
        panel in both cases)?  (Hint:  note the use of lines / hatchings in
        Romano’s painting, a characteristic of tempera).   Do you think any of
        the visual qualities you have described is related to the use of
        different media (tempera vs. oil)?  Note also the use of gold leaf in
        Romano’s painting.

        Romano:

        Dolci:
            Part of the level of detail can definitely be attributed to the oil
            paint used in this image.





Composition:
    1)  How are the figures posed?  Within a roughly similar pyramidal
        arrangement of figures, which composition employs more diagonal visual
        accents?  Describe these.

        Overall:
            The way they are posed is actually very similar.  The virgin and
            child are at about the same vertical level in Dolci's paintint
            while the virgin is highest in ROmano's, though this likely isn't
            significant.

            The way the virgin and child are together is very similar between
            the two as well - in both cases the virgin is placing her head
            softly against the child as she holds the baby and it walks in the
            center of the painting.

            Additionally, there is figure off to the left praying in each
            painting.  THough the figure is not the same in both, the relative
            positioning and action is almost exactly the same.


        Romano:
            Both of the main figures (virgin and child) are looking out towards
            teh auidience as is posing for a picture.

            Much more triangular with the virgin in the center extending to the
            top.

            The space int he middle of this painting is occupied by the virgin
            and the child.

        Dolci:
            The space in the middle of this painting is occupied by the gap
            between the virgin and the child.  THough both figures are overal
            centered in the image, the actual vertical center is not occupied
            by either.

    2)  How does each painter create the illusion of three-dimensional form and
        space?  Is one painting relatively more “three-dimensional” than the
        other? Why does it have this effect?

        Romano:
            From the placment of the figures - Jesus in front with mary in the
            back.  Her arm over the child and Jesus's arm wrapped around the
            back of the virgin's neck

        Dolci:
            The light source and shading, way edges are depicted


    3)  In which painting does the artist think of discrete, individual forms,
        and in which is there a greater “unity” of forms—i.e., so that the
        figures, limbs, etc. are more closely and dynamically intertwined with
        each other?

        Romano:

        Dolci:


    4)  In which painting is the composition completely contained by the frame,
        and in which does it seem to extend beyond the borders of the frame?
        How does this affect our perception of the relation between “our”
        space—the viewer’s space—and the space of the picture?

        Romano:
            Completley in the frame

        Dolci:
            The virgin as well as John the Baptist extent somewhat beyond the
            edge of the frame

            This depiction, along with the greater shading and level of detail
            in the painting create the illusion that the figures are more in
            our space than in some painting


Emotional Level:
    1)  How do the Virgin and Child relate to each other and to the viewer?
        You may focus on their poses, gestures, facial expressions, gazes, etc.

        Romano:

        Dolci:


    2)  Note the figure of St. John the Baptist in Dolci’s painting:  do you
        see any parallels in how it functions within the scene to the figure of
        the donor in Romano’s picture?

        Romano:

        Dolci:


    3)  In general, how do these paintings compare to each other in their
        expressive, emotional content?  Is one, relatively, more “restrained”
        and the other more intensely emotional than the other?

        Romano:

        Dolci:


Romano:

Dolci:

