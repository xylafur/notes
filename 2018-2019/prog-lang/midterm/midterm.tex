\documentclass[11pt]{article}
\usepackage{listings}
\usepackage{color}

\definecolor{dkgreen}{rgb}{0,0.6,0}
\definecolor{gray}{rgb}{0.5,0.5,0.5}
\definecolor{mauve}{rgb}{0.58,0,0.82}

\lstdefinestyle{MyC++} {
        frame=tb,
        language=C++,
        aboveskip=3mm,
        belowskip=3mm,
        showstringspaces=false,
        columns=flexible,
        basicstyle={\small\ttfamily},
        numbers=none,
        numberstyle=\tiny\color{gray},
        keywordstyle=\color{blue},
        commentstyle=\color{dkgreen},
        stringstyle=\color{mauve},
        breaklines=true,
        breakatwhitespace=true,
        tabsize=3
}
\lstdefinestyle{MyC} {
        frame=tb,
        language=C,
        aboveskip=3mm,
        belowskip=3mm,
        showstringspaces=false,
        columns=flexible,
        basicstyle={\small\ttfamily},
        numbers=none,
        numberstyle=\tiny\color{gray},
        keywordstyle=\color{blue},
        commentstyle=\color{dkgreen},
        stringstyle=\color{mauve},
        breaklines=true,
        breakatwhitespace=true,
        tabsize=3
}

\lstdefinelanguage{Asm} {
        keywords={mov, gtr, bif, ifgtr, goto, add, r1, r2}
}
\lstdefinestyle{MyAsm} {
        frame=tb,
        language=Asm,
        aboveskip=3mm,
        belowskip=3mm,
        showstringspaces=false,
        columns=flexible,
        basicstyle={\small\ttfamily},
        numbers=none,
        numberstyle=\tiny\color{gray},
        keywordstyle=\color{blue},
        commentstyle=\color{dkgreen},
        stringstyle=\color{mauve},
        breaklines=true,
        breakatwhitespace=true,
        tabsize=3
}
\lstset{language=C++,frame=none}
\lstset{language=C,frame=none}

\begin{document}

\title{James Richardson\\1555520\\Programming Languages Midterm}
\maketitle

\newpage
\section{C++ Functional to Postfix Notation}
For this question I am going to assume that this the functional expression
\it{add(5, multiply(4, 3))} \rm{would be given to me in list format as}
\it{["4", "3", "multiply", 5, "add"]}
\rm\\\\

\begin{lstlisting}[style=MyC++]
#include <string.h>
void convert_postfix(char ** input, unsigned int input_length, char * output){
    //We are assuming that the memory for output has been allocated before
    //being passed into this function and that its length is sufficient
    int index = 0, ii;
    for(ii = 0; ii < input_length; ii++){
        if(!strcmp(input[ii], "add")){
            output[index] = '+';
        }else if (!strcmp(input[ii], "multiply")){
            output[index] = '*';
        }else{
            strcpy(output + index, input[ii]);
            index += strlen(input[ii]);
        }
        //so we have a buffer in between each character or integer
        strcpy(output + index, " ");
        index += 1;
    }
}

int main(){
    /*
     *  Assuming that the parsing happens somewhere in here
     */
    char * postfix_notation = malloc(computed_length);
    convert_postfix(parsed_input, input_length, postfix_notation);
    printf("%s\n", postfix_notation);
    free(postfix_notation);
}
\end{lstlisting}

\newpage

\section{Mutation in C++}
Variable multation in c++ is enabled by default.  If we declare and define an
integer i then we can overwrite that value with another integer later.
We can use the const keyword to disable mutation of particular variables.

In passing variables into functions, mutation is generally disabled by default.
When you pass in an integer, float, bool or another simple data type then that
variable or value is passed by value to the function - which means that a copy
is made of the value and placed into the function stack.  If we want to pass in
a variable that can be mutated into a function, we can pass in that varaible
either as a pointer or by reference.


To discover the possible performance implications of this, let us consider a
function that takes in 2 matricies and then adds them.  This function will be
useless because it won't actually return anything for simplicity.  We will also
assume that the complier doesn't pull any funny business and actually
executes the code rather than ignore it for optimization.

You can't really pass arrays by value in c++, you can pass in structs by value
though.  I will create a matrix sctruct type for this example to allow us to
pass in the matrix by value.\\\\

\textbf{By Value Example}
\begin{lstlisting}[style=MyC++]
#define MATRIX_SIZE 100
typedef struct _matrix {
    int vals [MATRIX_SIZE] [MATRIX_SIZE];
} matrix;

void add_matricies(matrix a, matrix b){
    /*  Add the matricies
     */
}
\end{lstlisting}



To pass in these matricies by value, we have to copy each of the elements
into a new location on the function stack.  Instead of the size of the matrix
being 100, lets consider it to be n.  We are copying two matricies into the
fucntion by value, each of those matricies being of size $n * n$.  this means that
the time complexity to copy in both of these matricies into the fucntion stack
would be $2 * O(n^2)$, or in proper O() notation, $O(n^2)$\\

Now let us consider the situation where we pass in the matricies by reference
using a pointer\\

\textbf{By Reference Example}
\begin{lstlisting}[style=MyC++]
#define MATRIX_SIZE 100
typedef struct _matrix {
    int vals [MATRIX_SIZE] [MATRIX_SIZE];
} matrix;

void add_matricies(matrix * a, matrix * b){
    /*  Add the matricies
     */
}
\end{lstlisting}

In this case, we do not have to copy the values of each of these matricies.
Instead we just pass a reference to the locatin of each of the matricies to the
function - this is done by pushing those addresses onto the function stack.
Those operations are both order of 1.  Meaning that to pass in the matricies by
reference it takes O(1).. quite an improvment over the pass by value approach.




\newpage
\section{Conversion of Nested Loops into Assembly}
When considering a nested loop in both a higher level language and in assembly
language, 
the assembly code will actually be very simmilar to the higher level code (if we
assume we are using something lke c, c++, java, not python with their
iterators).
Though the assembly version of the loop will undoubtedly have many more
statements and will use goto statements rather than implicit looping
constructs.\\

A loop in assembly functions as follows (we are assuming the machine is
register based):
\begin{itemize}
    \item Check against the termination condition.  The result of this
        operation is stored ina special register.
    \item If the result of that condition was 1, branch to the end of the loop,
        otherwise continue executing at the next instruction.
    \item The next \textit{n} instructions will be the actual body of the loop,
        whatever that is.
    \item The very last instruction in the assembly loop construct is to jump
        back to the beginning of the loop.
\end{itemize}



Writing a nested loop in assembly is about the same as writing one in a higher
level language.  In the section that contains all of the code for the body of
our loop, we would simply insert another assembly loop construct.  This inner
assembly loop construct would have its own beginning address, check condition
and ending address.\\\\

\newpage
Below is an example of a nested loop written in C followed by the same nested
loop written in c++.\\

\textbf{Nested Loop in C}
\begin{lstlisting}[style=MyC]
int i, j;
for(i = 0; i < 10; i++){
    for(j=0; j < 10; j++){
        /*do something*/
    }
}
\end{lstlisting}

\textbf{Nested Loop in Psuedo Assembly}
\begin{lstlisting}[style=MyAsm]
    mov #0 r1
loop1:
    gtr r1 #10
    bif endloop1

    mov #0 r2
loop2:
    gtr r2 #10
    bif endloop2

    ; something code

    add r2 r2 1
    goto loop2

endloop2:
    add r1 r1 1
    goto loop1

endloop1:
    ; code after nested loop
\end{lstlisting}

\newpage
\section{The Influence of Older Languages on Modern Ones}

\begin{enumerate}
\item Lisp\\
    First to intruduce garbage collection
    \footnote{https://en.wikipedia.org/wiki/Lisp}


\item Stimula\\
    First Object oriented programming language.
    \footnote{https://en.wikipedia.org/wiki/Simula}


\item Algol 58\\
    Introduced the difference between comparison ans assignment operators
        := assignment, = equality

    \footnote{https://en.wikipedia.org/wiki/ALGOL_58#History}

\item Algol 60\\
    First programming language to use code blocks.  Algol 60 used 'begin' and
        'end' instead of the C curly braces though.
    \footnote{https://en.wikipedia.org/wiki/ALGOL_58#History}

\item Algol 60\\
    One of the first languages to introduce the dangling-else delima.
    \footnote{https://en.wikipedia.org/wiki/ALGOL_58#History}

\end{enumerate}

%num 5
\newpage
\section{Memory Management in Various Modern Languages}

\textbf{c++: list = vector}
\begin{itemize}
    \item allocation:
        you can create a list either the stack or the heap.
        Using the new keyword will allocate memory from the heap

    \item manipulation:
        you can use a pointer to manipulate or pass around the vector in c++,
        you can also pass it in using the reference k=word

        the list can also be passed around by value if you want to.

    \item clean up:
        If you allocate the list from the heap then you will have to manually
        clean it up.

    \item array declaration example:
        stack
            int arr [50];
        heap:
            int arr [50] = new int [50];

    \item list declaration example:
        stack:
            std:vector<int> my_stack_vec
        heap:
            std:vector<int> * myvec = new vector

    \item list insert elements example:

    \item list delete elements example:
\end{itemize}

\pagebreak
\textbf{Java: list = ArrayList}
\begin{itemize}
    \item allocation:
        Both ArrayLists and arrays are not simple data types, they cannot be
        stoed on the stack.  They must be allocated from the heap.

        Both are created using the new keyword

    \item manipulation:
        objects are passed into functions by reference implicitly, there is no
        way to pass them into a function by value.

    \item clean up:
        Java has garbage collection, which means as the programmer you do not
        free memory that you were granted from the heap.  WHen there is not
        enough free memory garabge collection will be invoked and release all
        unused heap memory.

        An ArrayList would definitely be more complicated to clean up due to
        the fact that it has more metadata than a regular array.  An array list
        will also have to constantly be cleaned up as its size is increased,
        the old array that was used to hold the previous data of the list will
        have to be freed.

    \item array declaration example:
        int [] arr = new int [50];

    \item list declaration example:

    \item list insert elements example:

    \item list delete elements example:
\end{itemize}


\pagebreak
\textbf{python: list = list , no arrays}
\begin{itemize}
    \item allocation:
        In python everything is on the heap.  Lists will therefore be on the
        heap just like everything else.  There are no arrays in python.

    \item manipulation:
        lists are passed by reference implicitly in python, there is no way to
        pass them by value.

    \item clean up:
        The garbage collector will clean up unused objects when the python vm
        runs out of memory or drops below a particular treshold.

    \item array declaration example:
        There are no arrays in python (well you can use numpy but its not a
        python builtin)

    \item list declaration example:
        L = []

    \item list insert elements example:
        L.insert(1)

    \item list delete elements example:
        L.pop() or L.remove(1)
\end{itemize}




%num 6
\newpage
\section{Requirements to Handle Floating Point and Integer Numbers in Python}
Currently our interpreter only supports integers, strings and boolean types.
In the case of our interpreter, we use a stack based virtual machine to compute
the actual program; the virtual machine executes bytecode that we created.  The
lexer/parser convert the python file into this bytecode and then this bytecode
is executed.\\

The way our Virtual Machine's instruction set is implemented, we have seperate
math instructions for the different value types: for instance we use
the pushi, pushi and addi commands to push an integer onto the stack and then add the
top two integers from the stack.\\

To add functionality that allows the use of floating point numbers we would
firstly have to properly tokenisze them.  Our parser would then need to be
updated to accept floating point numbers in the grammar.  This is the easiest
step, the regex would be something like:\\
\begin{lstlisting}
\d+\.\d*|\d*.\d+
\end{lstlisting}

When this is converted to the abstract syntax tree, the node for that floating
point value would be of type 'float' rather then integer.  When the abstract
syntax tree is converted into bytecode we would have to make sure to use the
proper instructions for the new type.\\

Type checking could be handled when converting from the AST.  Our expr nodes
can have a type, either integer or floating point.  Like most languages, we
would probably implement it such that if there is a floating point number
involved then the other number will also be treated as a float and then the
result will be a float.  So any expr node that uses a floating point number
node or another expr node that uses a float will be marked as such.  All of the
integers in these nodes will have to be converted into floating point
numbers.\\


To achieve this we would need to add a casting instruction to convert from the
format used to represent floating point numbers and the format used to
represent integer numbers.\\

This floating point number type would be implemented using the IEEE specified
scientific notation that other programming languages (and CPUs/ ALUs) use to
represent numbers. Our Virtual Machine would have to know how to manipulate a
floating point number, we would give the VM a special instruction so that it
knows we are using a floating point number.\\

Conversion would be a simple bit operation.  Floating point numbers are
represented something like:\\
\begin{lstlisting}
    <sign bit><exponent><significand>
\end{lstlisting}


When converting, we would have to decide on a convention, either rounding or
truncating.  For this question I will assume we use truncation.  So if we have
10.5, out number would become 10.0, then we would convert this to the integer
10.\\
We would convert this to an integer by copying over the significand exactly
into the integer and then shifting this number over based on the exponent of
the float. \footnote{https://stackoverflow.com/questions/7977265/what-happens-at-background-when-convert-int-to-float}

%num7
\newpage
\section{The Dangling Else Deleima}
I will use Java in answering this question rather than python.\\
The ambiguity comes from the fact that the curly braces are optional in the
java if-then-else block.  Using the curly braces to define where our if block
starts and begins will resolve ambiguity.  The following example shows this:\\


\textbf{Ambigious if else grammar}
\begin{lstlisting}[style=MyC++]
if ( <bool expr> ) <stmt> | if ( <bool expr> ) <stmt> else <stmt>
\end{lstlisting}



Using this grammar, let us consider the following example.\\

\begin{lstlisting}[style=MyC++]
if (true)
    System.out.println("true");
    if (false)
        System.out.println("False");
    else
        System.out.println("not False");
\end{lstlisting}



Based on the way that this is indented, we could assume that the else statment
goes with the inner if.  But this is not the way it will necesarrily be parsed.
Since whitespace is not considered in Java, this is equivalent:\\

\begin{lstlisting}[style=MyC++]
if (true)
    System.out.println("true");
    if (false)
        System.out.println("False");
else
    System.out.println("not False");
\end{lstlisting}


The output of this program could either be\\

\begin{lstlisting}
>>> true
\end{lstlisting}


if the else block is considered to be part of the outer if statment.  Or\\

\begin{lstlisting}
>>> true
>>> False
\end{lstlisting}




if the else is taken to be part of the inner statment.  Whatever convention the
compiler/ parser uses is what will determine the order.

If we add in curly braces to the grammar definition:

\textbf{Unambigious if else grammar}
\begin{lstlisting}[style=MyC++]
if ( <bool expr> ) { <stmts> } | if ( <bool expr> ) { <stmts> } else { <stmts> }
\end{lstlisting}


Considering the same example as before but with the new grammar:



\begin{lstlisting}[style=MyC++]
if (true){
    System.out.println("true");
    if (false){
        System.out.println("False");
    }
    else{
        System.out.println("not False");
    }
}
\end{lstlisting}



There is only one that this can be parsed based on the grammar, since we are
inside the <stmts> block of the outer if stmt.  The contents of this stmts
block is determined by the curly braces.  In python it is determined by the
whitespace



%num8
\newpage
\section{Minimal BNF for Lambda Calculus and Examples}

The minimal grammar given by the professor during the exam review was as
follows:\\
\begin{lstlisting}
        t : t                       #variable
          | lambda t.expression     #variable to expression
          | tt                      #application
          ;
\end{lstlisting}


\textbf{Example 1}
\begin{lstlisting}
        (λx.(λy.(λz.x + y + z)))(1)(2)(3)
        (λy.(λz.1 + y + z))(2)(3)
        (λz.1 + 2 + z)(3)
        1 + 2 + 3
\end{lstlisting}


\textbf{Example 2}
\begin{lstlisting}
\end{lstlisting}



%num9
\newpage
\section{Type Safety in Lambda Calculus}
Typed Lambda Calculus is implemented with only two simple data types, booleans
and integers.  Each of these can be considered two completley different sets
with no commonalities.  In the words of the professor, "once we are inside a
set we want to make sure that we don't end up in the other set. For type safety
we want to keep everything confined within its own set."\\

We know how integers interact with integers, and we know how booleans interact
with booleans.  By confining each of these types to their own sets and
ensuring that both of the sets are independent we can stay in the realms of
what we understand and know how to to.  Type safety is implemented by keeping
these two sets seperate.\\

There is an exception to this however: The is zero function is a way for us to
map out of the integer set and into the boolean set.\\

In regards to data types, we want to have a complete and sound definition.  We
can achieve this using three constructs:\\
\begin{enumerate}
    \item Type Expressions:\\
        T ::= ...\\
        Here we define what our types are

    \item Type Relations:\\
        t: T\\
        Here we define the relationship between a variable and its type

    \item Typing Rules Giving an Inductive Definition of t: T\\
        We define basic defitions for data types and then we have mechanisms
        for checking data types in general.
\end{enumerate}



%num10
\newpage
\section{Grammar for Arithmetic Expressions with Parenthesis}
The idea behind this is that anything encapselated in parenthesis should be at
the very bottom of our grammar defition.  Something being lower in the grammar
means that it has higher presedence than the things before it.

\begin{lstlisting}
expr1: <expr2> + <expr1>
     | <expr2> - <expr1>
     | <expr2>
     ;

expr2: <expr3> * <expr2>
     | <expr3> / <expr2>
     | <expr3>
     ;

expr3: - <expr4>
     | + <expr4>
     | <expr4>
     ;

expr4: ( <expr1> )
     | NUMBER
     ;
\end{lstlisting}


Recursive descent parsers work recursively (as implied by the name) while LR
parsers work using a stack.  Recursize descent parsers start from the beginning
of a file while LR parsers start from the end.

For an LR parser, tokens are pushed onto the stack.  At any point in time, the
parser can either shift a new token onto the stack or reduce.  In a properly
constructed grammar we always know if we will shift or reduce.  

\textbf{Recursive Descent Example}


\begin{lstlisting}
\end{lstlisting}

\textbf{LR Example}


\begin{lstlisting}
\end{lstlisting}



\end{document}
